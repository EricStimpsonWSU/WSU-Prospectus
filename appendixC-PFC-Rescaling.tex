% Create appendix with unnumbered section
%\section*{APPENDIX C - Vacancy PFC Rescaling Notes}
\appendixsection{C}{Vacancy PFC Rescaling Notes}
\setappendixlabel{C}

% Add appendix to toc at section level
\label{appendix:pfc-rescaling}

Adapted from Eq.~5 or Eq.~13 in \cite{huang2010phase}, the PFC model was derived from classical dynamical density functional theory (DDFT).  Eq.~13 is given here

\begin{align}\label{eq-pfc-vacancy-free-energy-ddft}
\frac{\mathcal{F}}{\rho_l k_B T} = \int \dpos \left[ (1 + n)\ln(1 + n) + \frac{1}{2}B^x n (2 R^2 \nabla^2 + R^4 \nabla^4) n + \frac{1}{2}{B_\text{l}}' n^2 + \frac{1}{3} \tilde{B} n^3 + \frac{1}{4} \tilde{B}_4 n^4 \right]
\end{align}
%
where the LHS is referred to as the ``free energy functional".  The phase field crystal (PFC) density $n$ is defined by

\begin{align}
n \equiv \frac{\rho - \densel}{\densel}, \quad \rho \ge 0 \Rightarrow n \ge -1.
\end{align}
%
where $\densel$ is a reference atomic density distribution function defined by the supercooled liquid state.  The various coefficients in the equation above are themselves defined in terms of $n$-point correlation functions $C^{(n)}$, or more specifically, the coefficients of their Fourier transforms $\hat{C}^{(n)}_i$ as,

\begin{align}
{B_\text{l}}' = B^l - 1 = \densel \hat{C}_0, \quad B^x = \densel \frac{(\hat{C}_2)^2}{4 \hat{C}_4}, \quad R = \sqrt{2 \hat{C}_4/\hat{C}_2}, \\
\tilde{B} = \densel^2 \hat{C}^{(3)}_0/2, \quad \tilde{B}_4 = \densel^3 \hat{C}^{(4)}_0/6, 
\end{align}
%
where the Fourier transform of the $n$-point correlation function $C^{(n)}(\pos_1,\ldots,\pos_n)$ is expanded as a power series in $k, k', k''$.  Example for $n=2$ is shown here

\begin{align}
C^{(2)} \overset{\mathcal{F}}{\rightarrow} \hat{C}^{(2)}(k) = - \hat{C}^{(2)}_0 + \hat{C}^{(2)}_2 k^2 - \hat{C}^{(2)}_4 k^4 + \cdots.
\end{align}

%\begin{align}
%C^{(2)} \overset{\mathcal{F}}{\rightarrow} \hat{C}^{(2)}(q) = - \hat{C}^{(2)}_0 + \hat{C}^{(2)}_2 q^2 - \hat{C}^{(2)}_4 q^4 + \cdots.
%\end{align}
%
%\begin{align}
%\hat{C}^{(2)}(q) \overset{\mathcal{F}^{-1}}{\rightarrow} = - \hat{C}^{(2)}_0 - \hat{C}^{(2)}_2 \nabla^2 - \hat{C}^{(2)}_4 \nabla^4 + \cdots = C^{(2)}.
%\end{align}
%
For $n=2$, the superscript $^{(2)}$ has been omitted for brevity such that $\hat{C}^{(2)}_i = \hat{C}_i$, while for $C^{(n)}$ with $n \ge 3$, the Fourier transform of $C$ has $n-1$ wave number arguments $k_1, \dots, k_{n-1}$, so expanding in a power series of $k$ would, in principle, require terms of the form $(k_1)^{a_1} \cdot (k_2)^{a_2} \cdots (k_{n-1})^{a_{n-1}}$.  However, in practice, only the coefficient of the $k_1 = k_2 = \cdots = k_{n-1} = 0$ term is actually used and is written $\hat{C}^{(n)}_0$.

The free energy functional in Eq.~\ref{eq-pfc-vacancy-free-energy-ddft} was renamed from $\mathcal{F}$ to $F$ and rewritten by factoring the gradient terms into a perfect square and renaming the leading coefficients on the polynomial terms as

\begin{align}
F(n) = \int d\vec{r} \left[ (1+n)\ln(1+n) + \frac{1}{2}B^x n (R^2\nabla^2 + 1)^2 n + \frac{1}{2}(-r)n^2 - \frac{1}{3}\tau_0 n^3 + \frac{1}{4}v_0 n^4 \right],
\end{align}
%
where

\begin{align}
-r = {B_\text{l}}' - B^x, \quad -\tau_0 = \frac{1}{2} {\densel}^2 {\hat{C}_0}^{(3)}, \quad v_0 = \frac{1}{6}{\densel}^3 {\hat{C}_0}^{(4)}.
\end{align}

The Cahn-Hilliard equation (conserved dynamics) describes the evolution of the system in time by a diffusion equation.

\begin{align}
\partialt n(\pos, t) &= M \vec{\nabla} \cdot \left[ (1 + n)\vec{\nabla} \frac{\delta F}{\delta n} \right] \simeq M \nabla^2 \mu \\
\Rightarrow \partialt n(\pos, t) &= D \nabla^2 \left[ \ln(1+n) + B^x (R^2 \nabla^2 + 1)^2 n - r n - \tau_0 n^2 + v_0 n^3 \right], \quad D = \frac{M}{k_B T}
\end{align}
%
with a slowly varying chemical potential $\mu = \frac{\delta F}{\delta n}$.

\subsection{Length \& Time Scale}
Introduce length scale $R$, time scale $R^2/D$, and reciprocal lattice spacing $q_0 = 1$, giving

\begin{subequations}
\begin{align}
F(n) &= \int d\vec{r} \left[ (1+n)\ln(1+n) + \frac{1}{2}B^x n (\nabla^2 + {q_0}^2)^2 n + \frac{1}{2}(-r)n^2 + \frac{1}{3} (-\tau_0) n^3 + \frac{1}{4}v_0 n^4 \right] \label{eq-pfc-nlnn-free-energy} \\
\partialt n(\pos, t) &= \nabla^2 \left[ \ln(1+n) + B^x (\nabla^2 + {q_0}^2)^2 n - r n - \tau_0 n^2 + v_0 n^3 \right]. \label{eq-pfc-nlnn-dndt}
\end{align}
\end{subequations}

\subsection{Eliminate \texorpdfstring{$v_0$}{v0}}
Define $n'$ by

\begin{align}
1 + n = \alpha(1 + n') \quad \Rightarrow n = \alpha n' + (\alpha - 1)
\end{align}
%
where $\alpha$ is a temporary parameter.  Then the change in PFC density, $dn/dt$ in Eq.~\ref{eq-pfc-nlnn-dndt}, becomes

\begin{subequations}
\begin{align}
\partialt n(\pos, t) &= \alpha \partialt n'(\pos, t) = \nabla^2 \left[ \textcolor{darkgreen}{\underbrace{ \textcolor{black}{ \cdots }}_{\text{keep only non-constant terms}}} \right] \\
&= \nabla^2 \left[ \ln (\alpha(1 + n')) + B^x (\nabla^2 + {q_0}^2)^2 (\alpha n' + (\alpha - 1)) +  \right. \nonumber \\
&\quad \left. -r (\alpha n' + (\alpha - 1)) - \tau_0 (\alpha n' + (\alpha - 1))^2 + v_0 (\alpha n' + (\alpha - 1))^3 \right] \nonumber \\
&= \nabla^2 \left[ \textcolor{darkred}{\ln \alpha} + \ln (1 + n') + \alpha B^x (\nabla^2 + {q_0}^2)^2 n' + \textcolor{darkred}{(\alpha - 1) B^x ({q_0}^2)^2} \right. \\
&\quad \left. -r (\alpha n' + \textcolor{darkred}{(\alpha - 1)}) - \tau_0 (\alpha^2 (n')^2 + 2 \alpha (\alpha - 1) n' + \textcolor{darkred}{(\alpha - 1)^2}) \right. \nonumber \\
&\quad \left. + v_0 (\alpha^3 (n')^3 + 3 \alpha^2 (\alpha-1) (n')^2 + 3 \alpha (\alpha-1)^2 n' + \textcolor{darkred}{(\alpha-1)^3}) \right] \nonumber \\
&= \nabla^2 \left[ \ln (1 + n') + \underbrace{\alpha B^x}_{\beta} (\nabla^2 + {q_0}^2)^2 n' - \underbrace{(\alpha r + 2 \alpha (\alpha - 1) \tau_0 - 3 \alpha (\alpha-1)^2 v_0)}_{r'} n' \right. \\
&\quad \left. - \underbrace{(\alpha^2 \tau_0 - 3 \alpha^2(\alpha - 1) v_0)(n')^2}_{{\tau_0}'} + \underbrace{\alpha^3 v_0}_{\underset{\Rightarrow \alpha = {v_0}^{-1/3}}{{v_0}' = 1}} (n')^3 ) \right] \nonumber \\
&= \nabla^2 \left[ \ln(1+n') + \beta(\nabla^2 + {q_0}^2)^2 n' - r' n' - {\tau_0} (n')^2 + (n')^3 \right],
\end{align}
\end{subequations}

with 

\begin{subequations}
\begin{align}
ta &= \alpha B^x = B^x {v_0}^{-1/3} \\
r' &= \alpha r + 2\alpha(\alpha - 1)\tau_0 - 3 \alpha (\alpha - 1)^2 v_0 \\
&= {v_0}^{-1/3} r + 2 {v_0}^{-1/3}({v_0}^{-1/3} - 1)\tau_0 - 3 {v_0}^{-1/3} ({v_0}^{-1/3} - 1)^2 v_0 \nonumber \\
&= {v_0}^{-1/3} r + 2 \tau_0 ({v_0}^{-2/3} - {v_0}^{-1/3}) - 3 (1 - 2{v_0}^{1/3} + {v_0}^{2/3})  \nonumber \\
&= 2 \tau_0 {v_0}^{-2/3} + (r - 2 \tau_0){v_0}^{-1/3} - 3 + 6 {v_0}^{1/3} - 3 {v_0}^{2/3} \nonumber \\
{\tau_0}' &= (\alpha^2 \tau_0 - 3 \alpha^2(\alpha - 1) v_0) = \tau_0 {v_0}^{-2/3} - 3 + 3 {v_0}^{1/3}
\end{align}
\end{subequations}

Dropping the primes and rescaling time $t \rightarrow {v_0}^{-1/3} t$ we are left with

\begin{subequations}
\begin{align}
F(n) &= \int d\vec{r} \left[ (1+n)\ln(1+n) + \frac{1}{2}\beta n (\nabla^2 + {q_0}^2)^2 n + \frac{1}{2}(-r)n^2 + \frac{1}{3}(-\tau_0) n^3 + \frac{1}{4} n^4 \right] \label{eq-pfc-nlnn-free-energy-rescaled} \\
\partialt n(\pos, t) &= \nabla^2 \left[ \ln(1+n) + \beta (\nabla^2 + {q_0}^2)^2 n - r n - \tau_0 n^2 + n^3 \right]. \label{eq-pfc-nlnn-dndt-rescaled}
\end{align}
\end{subequations}

%\section{Alternate Rescaling}

\subsection{Density Shift}\label{sec-zf-density-scale}
Substitute $n = \phi - 1$, giving $\phi=1+n'=(1+n)/\alpha=\rho/(\rho_l \alpha)$ so $\phi$ is proportional to atomic number density $\rho$, i.e., a rescaled $\rho$, and

\begin{align}
F(\phi) = \int d\vec{r} \left[ \phi \ln \phi + \frac{1}{2}\beta (\phi - 1) (\nabla^2 + {q_0}^2)^2 (\phi - 1) + \frac{1}{2}(-r)(\phi - 1)^2 + \frac{1}{3}(-\tau_0) (\phi - 1)^3 + \frac{1}{4} (\phi - 1)^4 \right]    
\end{align}
%
and

\begin{align}
\partialt \phi(\vec{r},t) &= \nabla^2 \left[ \ln \phi + \beta (\nabla^2 + {q_0}^2)^2 (\phi - 1) - r (\phi - 1) - \tau_0 (\phi - 1)^2 + (\phi - 1)^3 \right] \\
&= \nabla^2 
 [ \textcolor{darkgreen}{\underbrace{ \textcolor{black}{\ln \phi + \beta (\nabla^2 + {q_0}^2)^2 (\phi - 1) - r (\phi - 1) - \tau_0 (\phi - 1)^2 + (\phi - 1)^3}}_{\text{keep only non-constant terms}}} ] \\
&= \nabla^2 
 \left[ \ln \phi + \beta (\nabla^2 + {q_0}^2)^2 \phi - \textcolor{darkred}{\beta ({q_0}^2)^2} - r \phi + \textcolor{darkred}{r} - \tau_0 \phi^2 + 2 \tau_0 \phi - \textcolor{darkred}{\tau_0} + \phi^3 - 3 \phi^2 + 3 \phi - \textcolor{darkred}{1} \right] \\
&= \nabla^2 
 [ \ln \phi + \beta (\nabla^2 + {q_0}^2)^2 \phi + ( \textcolor{darkgray}{\underbrace{-r + 2 \tau_0 + 3}_{\epsilon}} ) \phi + ( \textcolor{darkgray}{\underbrace{-\tau_0 - 3}_{g}} ) \phi^2 + \phi^3 ] \\
\Rightarrow \partialt \phi(\vec{r},t) &= \nabla^2 
 \left[ \ln \phi + \beta (\nabla^2 + {q_0}^2)^2 \phi + \epsilon \phi + g \phi^2 + \phi^3 \right],
 \end{align}
%
with

\begin{align}
\epsilon = -(r - 2 \tau_0' - 3), \quad g = -(\tau_0' + 3).
\end{align}
%
Substituting primed coefficients gives

\begin{subequations}
\begin{align}
\epsilon &= -(r - 2 \tau_0' - 3) \\
&= (3 v_0 + 2 \tau_0 - r){v_0}^{-1/3} \nonumber \\
g &= -(\tau_0' + 3) \\
&= -(\tau_0 + 3 v_0) {v_0}^{-2/3}   \nonumber
\end{align}
\end{subequations}
%
and

\begin{align}\label{eq-pfc-vacancy-free-energy-final}
F(\phi) &= \int d\vec{r} \left[ \phi \ln \phi + \frac{1}{2}\beta \phi (\nabla^2 + {q_0}^2)^2 \phi + \frac{1}{2}\epsilon \phi^2 + \frac{1}{3}g \phi^3 + \frac{1}{4} \phi^4 \right]. \\
\partialt \phi(\vec{r},t) &= \nabla^2 
 \left[ \ln \phi + \beta (\nabla^2 + {q_0}^2)^2 \phi + \epsilon \phi + g \phi^2 + \phi^3 \right]
\end{align}
%
for conserved field $\phi$.

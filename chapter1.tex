\section{Introduction}
\label{chap:Introduction}

\subsection{Computational Models for Atomic Condensed Matter Physics}

\begin{itemize}
    \item Molecular dynamics description
    \item Density field theories
    \item Classical density field theory
    \item Phase field crystal
\end{itemize}

\begin{figure}[!htbp]
    \centering
    \includegraphics[width=0.75\linewidth]{fig/fig-md-field-theory.pdf}
    \caption{Molecular Dynamics (MD) vs Field Theory Description} MD simulations, as illustrated on the left, treat atoms as point particles in a $6N$ vector space with position and momentum vectors $(\{\mathbf{q}_i\},\{\mathbf{p}_i\})$, $i = 1, 2, \ldots, N$.  The trajectories of atoms in an MD simulation are calculated by time-stepping Newtonian equations of motion [i.e. $\mathbf{q}_i(t + \Delta t) = \mathbf{q}_i(t) + \mathbf{p}_i/m \cdot \Delta t$ and $\mathbf{p}_i(t + \Delta t) = \mathbf{p}_i(t) - \nabla U \cdot \Delta t$], with typical microscopic timescales on the order of $\tau_{\mu} = 10^{-12}$ s, making it impractical for observing diffusion.  Field theories, as illustrated on the right, integrate out the microscopic motion of the atoms to arrive at a number density field $\rho$, discretized over a regular lattice.  The change in the density field is given by $\rho(t + \Delta t) = \rho(t) + D \nabla^2 [\delta F / \delta \rho(t)] \cdot \Delta t$, where $D$ is a diffusion coefficient.
    \label{fig:WSUlogo}
\end{figure}

\clearpage

\subsection{Phase Field Crystal Model Variants}

\begin{itemize}
    \item Standard PFC Model
    \item Multi-mode PFC
    \item Coupled fields (binaries, substrate)
    \item Amplitude expansion
    \item Structure PFC (XPFC)
\end{itemize}

\subsection{Successes of PFC}

\begin{itemize}
    \item Interface energy
    \item Topological defect energy
    \item Epitaxial crystal growth
    \item Structure PFC (XPFC)
    \item Vacancy diffusion?
\end{itemize}

%\begin{table}[!h]
%    \centering	
%    \bgroup
%    \def\arraystretch{1.00}
%    \begin{tabular}{| c | c | c | c |}
%          \hline			
%          Resource & Website & What's it for? \\ \hline \hline \cline{1-3}
%          Academic Success Center & https://success.wayne.edu/ & Academic Success!\\ \hline
%          Campus Health Center & http://health.wayne.edu/ & Health! \\ \hline
%          WRT Zone & http://www.clas.wayne.edu/writing/ & Writing Help!\\ \hline
%    \end{tabular}
%    \egroup
%    \caption{Wayne State University Resources}
%    \label{tab:WSUresources}
%\end{table}